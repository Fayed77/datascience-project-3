% Options for packages loaded elsewhere
\PassOptionsToPackage{unicode}{hyperref}
\PassOptionsToPackage{hyphens}{url}
%
\documentclass[
]{article}
\usepackage{lmodern}
\usepackage{amssymb,amsmath}
\usepackage{ifxetex,ifluatex}
\ifnum 0\ifxetex 1\fi\ifluatex 1\fi=0 % if pdftex
  \usepackage[T1]{fontenc}
  \usepackage[utf8]{inputenc}
  \usepackage{textcomp} % provide euro and other symbols
\else % if luatex or xetex
  \usepackage{unicode-math}
  \defaultfontfeatures{Scale=MatchLowercase}
  \defaultfontfeatures[\rmfamily]{Ligatures=TeX,Scale=1}
\fi
% Use upquote if available, for straight quotes in verbatim environments
\IfFileExists{upquote.sty}{\usepackage{upquote}}{}
\IfFileExists{microtype.sty}{% use microtype if available
  \usepackage[]{microtype}
  \UseMicrotypeSet[protrusion]{basicmath} % disable protrusion for tt fonts
}{}
\makeatletter
\@ifundefined{KOMAClassName}{% if non-KOMA class
  \IfFileExists{parskip.sty}{%
    \usepackage{parskip}
  }{% else
    \setlength{\parindent}{0pt}
    \setlength{\parskip}{6pt plus 2pt minus 1pt}}
}{% if KOMA class
  \KOMAoptions{parskip=half}}
\makeatother
\usepackage{xcolor}
\IfFileExists{xurl.sty}{\usepackage{xurl}}{} % add URL line breaks if available
\IfFileExists{bookmark.sty}{\usepackage{bookmark}}{\usepackage{hyperref}}
\hypersetup{
  pdftitle={Project 1},
  pdfauthor={Fayed Khan Barno},
  hidelinks,
  pdfcreator={LaTeX via pandoc}}
\urlstyle{same} % disable monospaced font for URLs
\usepackage[margin=1in]{geometry}
\usepackage{color}
\usepackage{fancyvrb}
\newcommand{\VerbBar}{|}
\newcommand{\VERB}{\Verb[commandchars=\\\{\}]}
\DefineVerbatimEnvironment{Highlighting}{Verbatim}{commandchars=\\\{\}}
% Add ',fontsize=\small' for more characters per line
\usepackage{framed}
\definecolor{shadecolor}{RGB}{248,248,248}
\newenvironment{Shaded}{\begin{snugshade}}{\end{snugshade}}
\newcommand{\AlertTok}[1]{\textcolor[rgb]{0.94,0.16,0.16}{#1}}
\newcommand{\AnnotationTok}[1]{\textcolor[rgb]{0.56,0.35,0.01}{\textbf{\textit{#1}}}}
\newcommand{\AttributeTok}[1]{\textcolor[rgb]{0.77,0.63,0.00}{#1}}
\newcommand{\BaseNTok}[1]{\textcolor[rgb]{0.00,0.00,0.81}{#1}}
\newcommand{\BuiltInTok}[1]{#1}
\newcommand{\CharTok}[1]{\textcolor[rgb]{0.31,0.60,0.02}{#1}}
\newcommand{\CommentTok}[1]{\textcolor[rgb]{0.56,0.35,0.01}{\textit{#1}}}
\newcommand{\CommentVarTok}[1]{\textcolor[rgb]{0.56,0.35,0.01}{\textbf{\textit{#1}}}}
\newcommand{\ConstantTok}[1]{\textcolor[rgb]{0.00,0.00,0.00}{#1}}
\newcommand{\ControlFlowTok}[1]{\textcolor[rgb]{0.13,0.29,0.53}{\textbf{#1}}}
\newcommand{\DataTypeTok}[1]{\textcolor[rgb]{0.13,0.29,0.53}{#1}}
\newcommand{\DecValTok}[1]{\textcolor[rgb]{0.00,0.00,0.81}{#1}}
\newcommand{\DocumentationTok}[1]{\textcolor[rgb]{0.56,0.35,0.01}{\textbf{\textit{#1}}}}
\newcommand{\ErrorTok}[1]{\textcolor[rgb]{0.64,0.00,0.00}{\textbf{#1}}}
\newcommand{\ExtensionTok}[1]{#1}
\newcommand{\FloatTok}[1]{\textcolor[rgb]{0.00,0.00,0.81}{#1}}
\newcommand{\FunctionTok}[1]{\textcolor[rgb]{0.00,0.00,0.00}{#1}}
\newcommand{\ImportTok}[1]{#1}
\newcommand{\InformationTok}[1]{\textcolor[rgb]{0.56,0.35,0.01}{\textbf{\textit{#1}}}}
\newcommand{\KeywordTok}[1]{\textcolor[rgb]{0.13,0.29,0.53}{\textbf{#1}}}
\newcommand{\NormalTok}[1]{#1}
\newcommand{\OperatorTok}[1]{\textcolor[rgb]{0.81,0.36,0.00}{\textbf{#1}}}
\newcommand{\OtherTok}[1]{\textcolor[rgb]{0.56,0.35,0.01}{#1}}
\newcommand{\PreprocessorTok}[1]{\textcolor[rgb]{0.56,0.35,0.01}{\textit{#1}}}
\newcommand{\RegionMarkerTok}[1]{#1}
\newcommand{\SpecialCharTok}[1]{\textcolor[rgb]{0.00,0.00,0.00}{#1}}
\newcommand{\SpecialStringTok}[1]{\textcolor[rgb]{0.31,0.60,0.02}{#1}}
\newcommand{\StringTok}[1]{\textcolor[rgb]{0.31,0.60,0.02}{#1}}
\newcommand{\VariableTok}[1]{\textcolor[rgb]{0.00,0.00,0.00}{#1}}
\newcommand{\VerbatimStringTok}[1]{\textcolor[rgb]{0.31,0.60,0.02}{#1}}
\newcommand{\WarningTok}[1]{\textcolor[rgb]{0.56,0.35,0.01}{\textbf{\textit{#1}}}}
\usepackage{graphicx,grffile}
\makeatletter
\def\maxwidth{\ifdim\Gin@nat@width>\linewidth\linewidth\else\Gin@nat@width\fi}
\def\maxheight{\ifdim\Gin@nat@height>\textheight\textheight\else\Gin@nat@height\fi}
\makeatother
% Scale images if necessary, so that they will not overflow the page
% margins by default, and it is still possible to overwrite the defaults
% using explicit options in \includegraphics[width, height, ...]{}
\setkeys{Gin}{width=\maxwidth,height=\maxheight,keepaspectratio}
% Set default figure placement to htbp
\makeatletter
\def\fps@figure{htbp}
\makeatother
\setlength{\emergencystretch}{3em} % prevent overfull lines
\providecommand{\tightlist}{%
  \setlength{\itemsep}{0pt}\setlength{\parskip}{0pt}}
\setcounter{secnumdepth}{-\maxdimen} % remove section numbering

\title{Project 1}
\author{Fayed Khan Barno}
\date{1/18/2021}

\begin{document}
\maketitle

\begin{Shaded}
\begin{Highlighting}[]
\KeywordTok{library}\NormalTok{(tidyverse)}
\end{Highlighting}
\end{Shaded}

\begin{verbatim}
## -- Attaching packages --------------------------------------- tidyverse 1.3.0 --
\end{verbatim}

\begin{verbatim}
## v ggplot2 3.3.3     v purrr   0.3.4
## v tibble  3.0.4     v dplyr   1.0.2
## v tidyr   1.1.2     v stringr 1.4.0
## v readr   1.4.0     v forcats 0.5.0
\end{verbatim}

\begin{verbatim}
## -- Conflicts ------------------------------------------ tidyverse_conflicts() --
## x dplyr::filter() masks stats::filter()
## x dplyr::lag()    masks stats::lag()
\end{verbatim}

\begin{Shaded}
\begin{Highlighting}[]
\NormalTok{Island<-}\StringTok{ }\KeywordTok{read_csv}\NormalTok{(}\StringTok{"C:/Users/Fayed/Desktop/Assignments/CMSC 205/DATA.csv"}\NormalTok{)}
\end{Highlighting}
\end{Shaded}

\begin{verbatim}
## 
## -- Column specification --------------------------------------------------------
## cols(
##   first_name = col_character(),
##   last_name = col_character(),
##   age = col_double(),
##   Happy_Sad_group = col_character(),
##   Dosage = col_double(),
##   Drug = col_character(),
##   Mem_Score_Before = col_double(),
##   Mem_Score_After = col_double(),
##   Diff = col_double()
## )
\end{verbatim}

\hypertarget{effects-of-anti-anxiety-medication-on-memory-another-intrusive-side-effect-of-anti-anxiety-medication}{%
\paragraph{Effects of Anti-anxiety Medication on Memory: Another
Intrusive Side-Effect of Anti-anxiety
Medication?}\label{effects-of-anti-anxiety-medication-on-memory-another-intrusive-side-effect-of-anti-anxiety-medication}}

According to Aspenridge Recovery anxiety disorders are the most common
mental illness in the United States, affecting over 40 million adults
every year. While being beneficial to some, it has been well debated
whether their debilitating side effects are worth their benefit.
Benzodiazepine is a class of anti-anxiety drugs widely used to treat
anxiety, panic attacks, depression and etc. According to
psychiatryadvisor.com approximately 30.6 million adults in the US used
benzodiazepine in 2019. According to Wikipedia, in 2017 10,684 died
because of benzodiazepine related incidents and the numbers are still
skyrocketing. Hence, it is important to evaluate the extent of the side
effects of Benzodiazepines. The figure below contain data regarding
trials done on individuals of a range of ages with two drugs belonging
to the Benzodiazepine class: Alprazolam and Triazolam. Sugar pills were
also introduced to the group as control, but the group was told that
they were given medication. The difference in time to remember two
specific types of memory: happy or sad, was also recorded to assess the
Drug's effect on memory.

\begin{Shaded}
\begin{Highlighting}[]
  \KeywordTok{ggplot}\NormalTok{(}\DataTypeTok{data=}\NormalTok{Island, }\KeywordTok{aes}\NormalTok{(}\DataTypeTok{x=}\NormalTok{age)) }\OperatorTok{+}\StringTok{ }
\StringTok{  }
\StringTok{  }\KeywordTok{geom_bar}\NormalTok{(}\KeywordTok{aes}\NormalTok{(}\DataTypeTok{y=}\NormalTok{Diff), }\DataTypeTok{fill=}\StringTok{"white"}\NormalTok{, }\DataTypeTok{color=}\StringTok{"red"}\NormalTok{, }\DataTypeTok{stat=}\StringTok{"identity"}\NormalTok{,}\DataTypeTok{width=}\FloatTok{0.2}\NormalTok{)}\OperatorTok{+}

\KeywordTok{facet_grid}\NormalTok{(Happy_Sad_group }\OperatorTok{~}\StringTok{ }\NormalTok{Drug)}\OperatorTok{+}

\KeywordTok{geom_hline}\NormalTok{(}\KeywordTok{aes}\NormalTok{(}\DataTypeTok{yintercept =} \KeywordTok{mean}\NormalTok{(Diff)))}\OperatorTok{+}
\StringTok{  }
\StringTok{  }\KeywordTok{ylab}\NormalTok{(}\StringTok{"Difference in Time/s, Before Subtracted from After(Less is better)"}\NormalTok{)}\OperatorTok{+}
\StringTok{  }
\StringTok{  }\KeywordTok{xlab}\NormalTok{(}\StringTok{"Age of subject"}\NormalTok{)}\OperatorTok{+}

\StringTok{  }\KeywordTok{geom_text}\NormalTok{(}\DataTypeTok{x=}\DecValTok{60}\NormalTok{, }\DataTypeTok{y=}\DecValTok{50}\NormalTok{, }\DataTypeTok{label=}\StringTok{"Mean Difference line"}\NormalTok{, }\DataTypeTok{color=}\StringTok{"black"}\NormalTok{)}\OperatorTok{+}
\StringTok{  }
\StringTok{  }\KeywordTok{labs}\NormalTok{(}\DataTypeTok{title=}\StringTok{"Effect of Anti-anxiety Drugs on Memory:"}\NormalTok{)}\OperatorTok{+}
\StringTok{  }\KeywordTok{geom_curve}\NormalTok{(}\DataTypeTok{x=}\DecValTok{70}\NormalTok{, }\DataTypeTok{xend=}\DecValTok{85}\NormalTok{, }\DataTypeTok{y=}\DecValTok{45}\NormalTok{, }\DataTypeTok{yend=}\DecValTok{3}\NormalTok{, }\DataTypeTok{arrow=}\KeywordTok{arrow}\NormalTok{(}\DataTypeTok{length=}\KeywordTok{unit}\NormalTok{(}\FloatTok{0.3}\NormalTok{, }\StringTok{"cm"}\NormalTok{)), }\DataTypeTok{curvature=}\FloatTok{0.2}\NormalTok{) }
\end{Highlighting}
\end{Shaded}

\includegraphics{Project1_Fayed_Barno_files/figure-latex/unnamed-chunk-3-1.pdf}
Initially, it is apparent the drugs do make a significant change in the
cognitive function of the subjects. The subjects take more time to
remember both memories as signified by the mean difference line which
has a value greater than 0. This means the subjects take more time to
remember both type of memories, hence the medicine is significantly
altering cognitive processes in the brain and is suppressing memories.

Also, an interesting observation is that the bars on the top half of the
figure are taller than the lower half, which means the subjects take
lesser time to remember sad memories. This is true for all 6 graphs
hence; this is not a side effect of the medication. According webmd.com
this is an intrinsic nature of human beings, or in other words a result
of evolution. It is beneficial in the sense that our brain will be
focused on potentially threatening information (or sad memories).
Surprisingly, patients that were given Sugar pills performed better in
the memory test, most actually scored better (or less) contrary to ones
that were actually given doses of the medicine. This reflects the
brain's ability to both amplify or suppress different cognitive
processes, and how powerful of a role can placebo play in inducing or
repelling anxiety.

\end{document}
